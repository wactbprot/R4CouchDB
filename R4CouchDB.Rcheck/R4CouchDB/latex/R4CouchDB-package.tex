\HeaderA{R4CouchDB-package}{Collection of R functions for CouchDB access}{R4CouchDB.Rdash.package}
\aliasA{R4CouchDB}{R4CouchDB-package}{R4CouchDB}
\keyword{package}{R4CouchDB-package}
\begin{Description}\relax
The R4CouchDB package provides a collection of functions for
basic database and document management operations such as add and
delete.
\end{Description}
\begin{Details}\relax
\Tabular{ll}{
Package: & R4CouchDB\\
Type: & Package\\
Version: & 0.1\\
Date: & 2009-08-29\\
License: & BSD\\
LazyLoad: & yes\\
}
Every \code{cdbFunction()} gets and emits a list containing the connection
set up. Internally this \code{list()} is called \code{cdb}. The result
of  requests, the CouchDB answers are stored in the
\code{cdb\$res}.

The function \code{cdbIni()} initially provides
\code{cdb}. However \code{a <- list(serverName="localhost",
  port="5984")} is sufficient for calling \code{cdbGetUuids(a)}.

The Rd-examples in this package are a kind of fake. Every
function tests the needed list entries before request execution. Since
it is not possible to provide a default \code{cdb\$serverName} all
function calls in the example- sections set a \code{cdb\$error <- "no
  cdb\$serverName"}, no http- request is executed.

The http connection and the json transformations are done with Duncan
Temple Langs RCurl and RJSON.
\end{Details}
\begin{Author}\relax
wact.b.prot

Maintainer: thsteinbock@web.de
\end{Author}
\begin{References}\relax
\url{  http://www.omegahat.org/RCurl/}
\url{  http://www.omegahat.org/RJSONIO/}
\url{  http://couchdb.apache.org/}
\end{References}
\begin{Examples}
\begin{ExampleCode}
## get the list structure
cdb <- cdbIni()
## this structure includes:
cdb
## the command
cdb <- cdbListDB(cdb)
## yields
cdb$error
\end{ExampleCode}
\end{Examples}

