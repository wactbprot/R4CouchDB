\HeaderA{cdbAddDoc}{Generates a new document}{cdbAddDoc}
\keyword{misc}{cdbAddDoc}
\begin{Description}\relax
This function adds a new document to an already existing database
\end{Description}
\begin{Usage}
\begin{verbatim}
cdbAddDoc(cdb)
\end{verbatim}
\end{Usage}
\begin{Arguments}
\begin{ldescription}
\item[\code{cdb}] The list \code{cdb} only has to contain a
\code{cdb\$dataList} which is not an empty \code{list()}.

\end{ldescription}
\end{Arguments}
\begin{Details}\relax
This function is called addDoc (which means add a new
document). Therefore the
\code{cdb\$id} is requested fresh for every document to add.
\end{Details}
\begin{Value}
\begin{ldescription}
\item[\code{cdb}] The result is stored in \code{cdb\$res}

\end{ldescription}
\end{Value}
\begin{Author}\relax
wact.b.prot
thsteinbock@web.de
\end{Author}
\begin{SeeAlso}\relax
\code{cdbGetDoc()}
\end{SeeAlso}

